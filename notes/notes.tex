\documentclass[12pt]{amsart}

\usepackage{amsmath, amsthm, fullpage}

\newcommand{\CC}{\mathbb{C}}
\newcommand{\NN}{\mathbb{N}}
\newcommand{\QQ}{\mathbb{Q}}
\newcommand{\RR}{\mathbb{R}}
\newcommand{\ZZ}{\mathbb{Z}}

\begin{document}
\setlength\parindent{0em}
\setlength\parskip{2em}
\renewcommand{\baselinestretch}{1.5}

\section{Algebra}

$*$ A \emph{complex number}, $z$, is an expression of the form
\[
    z=a+bi,\quad \text{where}\quad a,b\in\RR.
\]
Set
\[
\Re(z)=a \quad\text{(real part)},\quad
\Im(z)=b  \quad\text{(imaginary part)}.
\]
Write $\CC$ for the set of all complex numbers:
\[
    \CC = \{a+bi : a,b\in \RR\}.
\]
We view $\RR$ as a subset of $\CC$ by identifying $a+0i\in\CC$ and $a\in\RR$.
A complex number of the form $0+bi$ is called \emph{purely imaginary}.


Two complex numbers are equal when their real are equal and their imaginary parts are equal:
\[
    w = z \Longleftrightarrow \Re(w)=\Re(z)\quad\text{and}\quad \Im(w)=\Im(z)
\]




$*$ Add and subtract complex numbers ``as usual'':
\[
    (a+ bi) \pm (c + di) = (a+c)\pm (b+d)i
\]
Equivalently,
\[
    \Re(z\pm w) = \Re(z) \pm \Re(w),\quad \Im(z\pm w)=\Im(z) \pm \Im(w).
\]

$*$ Multiply complex numbers ``as usual'', subject to the extra rule $i^2=-1$:
\[
    (a+ bi) (c + di) = (ac - bd) + (ad + bc)i
\]
Equivalently,
\[
    \Re(zw) = \Re(z)\Re(w) - \Im(z)\Im(w),\quad
    \Im(zw) = \Re(z)\Im(w) + \Im(z)\Re(w).
\]

$*$ Here's a formula for the $z/w$, $w\neq 0$. Do not memorize it! (See below.)
\[
        \frac{a+ bi}{c + di} = \frac{(ac + bd)}{a^2+b^2} + \frac{-ad + bc}{a^2+b^2}i
\]
We get reciprocals as a special case

$*$ Define the \emph{complex conjugate}, $\bar z$, of $z$ by
\[
    \bar z = a - bi.
\]
Equivalently,
\[
    \Re(\bar z) = \Re(z),\quad \Im(\bar z) = -\Im(z).
\]
Observe that
\[
    z\in\RR \Longleftrightarrow z=\bar z,\quad z+\bar z=2\Re(z)\in\RR,\quad z-\bar z=2i\Im(z)\in\RR i
\]
We have:
\[
    \overline{z \pm w} = \bar z\pm \bar w,\quad
    \overline{zw} = \bar z\bar w,\quad
    \overline{z/w} = \bar z/\bar w,\quad
    \overline{1/z} = 1/\bar z,\quad
    z\bar z = a^2+b^2
\]
Complex conjugates appear in the quadratic formula for negative discriminants.
If $a,b,c\in\RR$ and $D := b^2-4ac <0$, then the roots of $az^2 + bz + c=0$ are
\[
    z=\frac{-b + \sqrt{D}i}{2a}\quad\text{and}\quad \bar z =\frac{-b-\sqrt Di}{2a}.
\]

\textbf{Theorem:} Every quadratic polynomial with real coefficients has two roots in $\CC$, counted with multiplicity.

$*$ Extend the absolute value function from $\RR$ to $\CC$ by setting
\[
    |z| = \sqrt{a^2+b^2}.
\]
$|z|$ is also called the \emph{modulus} of $z$.
Notice that
\[
z\bar z = |z|^2.
\]
We have $|z|\geq 0$, with equality if and only if $z=0$. Also,
\[
    |z+w|\leq |z| + |w|\quad \text{triangle inequality},\quad |zw|=|z||w|.
\]
The property $|zw|=|z||w|$ is equivalent to the identity
\[
    (a^2+b^2)(c^2+d^2)=(ac-bd)^2 + (ad+bc)^2.
\]
Thus, a product of sums of squares is a sum of squares.

Observe that if $w\neq 0$, then
\[
    \frac wz = \frac{w\bar z}{z\bar z} = \frac{w\bar z}{|z|^2} = \frac{w\bar z}{a^2+b^2}.
\]
To divide $z$ by $w$, just multiply numerator and denominator by $\bar w$ and then follow your nose.
Reciprocals are easy:
\[
    \frac1z = \frac{\bar z}{|z|^2} = \frac{\bar z}{a^2+b^2}.
\]
It's useful to note that
\[
    |z|=1\Longleftrightarrow \frac1z=\bar z.
\]

$3^2+4^2=5^2$,  $5^2+12^2=13^2$, $33^2 + 56^2 = 65^2$


\section{Geometry}

\end{document}