\documentclass[12pt]{exam}

\usepackage{fullpage, amsmath, amssymb, amsthm}

\renewcommand{\partshook}{\setlength\itemsep{1em}}
\newcommand{\disp}{\displaystyle}
\newcommand{\CC}{\mathbb{C}}
\newcommand{\RR}{\mathbb{R}}
\renewcommand{\Re}{\operatorname{Re}}
\renewcommand{\Im}{\operatorname{Im}}
\DeclareMathOperator{\Log}{Log}
\DeclareMathOperator{\Arg}{Arg}
\DeclareMathOperator{\sign}{sign}
\begin{document}

\section*{MATH 307 --- Worksheet \#2 }

\begin{questions}
    \setlength\itemsep{1.25em}
    \setlength\parskip{1em}

    \question
    Let $\sqrt{\cdot}$ denote the branch of the square root defined by
    \[
        \sqrt {re^{i\theta}} = \sqrt re^{i\theta/2},\quad \theta\in [0,2\pi)
    \]
    For which $z$ does the identity $\sqrt{z^2}=z$ hold?

    \begin{solution}
        Write $z=re^{i\theta}$, $\theta\in[0,2\pi)$. Then
        \[
            z^2 = r^2e^{i(2\theta)}
        \]
        If $0\leq \theta < \pi$, then $0\leq 2\theta < 2\pi$ and
        \[
            \sqrt {z^2} = \sqrt{r^2e^{i(2\theta)}} = re^{i(2\theta)/2} = re^{i\theta}=z.
        \]
        If, however, $\pi\leq \theta < 2\pi$, then $2\pi\leq \theta < 4\pi$,
        in which case
        \[
            \sqrt {z^2} = \sqrt{r^2e^{i(2\theta)}} = \sqrt{r^2e^{i(2\theta-2\pi)}} =re^{i(2\theta-2\pi)/2} = re^{i\theta}e^{-i\pi}=-z.
        \]
        Thus, $\sqrt{z^2}=\sqrt z$ if and only if $\theta\in [0,\pi)$.
    
    \end{solution}
    \question
    Find all values.
    \begin{parts}
        \item $\log 1$
        \begin{solution}
            \[
                \log 1 = \log |1| + i\arg 1 = 2k\pi i
            \]
        \end{solution}

        \item $\log (1+i)$
        \begin{solution}
            \[
                \log (1+i) = \log |1+i| + i\arg(1+i) = \log\sqrt 2 + i\left(\frac\pi 4 + 2k\pi\right)
            \]
        \end{solution}

        \item $(1+i)^{1+i}$
        \begin{solution}
            \begin{align*}
                (1+i)^{1+i} &= e^{(1+i)\log(1+i)}\\
                &= e^{(1+i)\left\{\log\sqrt 2 + i\left(\frac\pi 4 + 2k\pi\right)\right\}}\\
                &= e^{\log\sqrt2-\left(\frac\pi4+2k\pi\right)}e^{i(\log\sqrt2 + \frac\pi4 + 2k\pi)}\\
                &= \sqrt2e^{-\left(\frac\pi4+2k\pi\right)}
                \left\{\cos\left(\log\sqrt2 + \frac\pi4 + 2k\pi\right) + i\sin\left(\log\sqrt2 + \frac\pi4 + 2k\pi\right)\right\}
            \end{align*}
        \end{solution}
    \end{parts}

    \question Let $z=re^{i\theta}$. Express all values of $z^z$ in the form $x+iy$.

    \begin{solution}
        Write $z=re^{i\theta}$.
        \begin{align*}
            z^z &= e^{z\log z}\\
            &= e^{(r\cos\theta+ir\sin\theta)(\log r + i(\theta+2k\pi))}\\
            &= e^{r\cos\theta\log r - (\theta + 2k\pi) r\sin\theta}e^{i(r(\theta+2k\pi)\cos\theta + r\log r\sin\theta)}\\
            &= r^{r\cos\theta}e^{- (\theta+2k\pi) r\sin\theta}\left\{\cos(r(\theta+2k\pi)\cos\theta + r\log r\sin\theta)\right.\\
            &\qquad\qquad+ \left.i\sin(r(\theta+2k\pi)\cos\theta + r\log r\sin\theta)\right\}
        \end{align*}
    \end{solution}

    \question
    Compute the limit or argue that it don't exist.
    \begin{parts}
        \part $\displaystyle{\lim_{x\to\infty}e^{x+iy}}$ (fixed $y$)

        \begin{solution}
            \[
            \lim_{x\to\infty}e^{x+iy}= e^{iy}\lim_{x\to\infty} e^x = \infty
            \]
        \end{solution}
        \part $\displaystyle{\lim_{x\to-\infty}e^{x+iy}}$ (fixed $y$)
        \begin{solution}
            \[
            \lim_{x\to-\infty}e^{x+iy}= e^{iy}\lim_{x\to-\infty} e^x = 0
            \]
        \end{solution}

        \part $\displaystyle{\lim_{y\to\infty}e^{x+iy}}$ (fixed $x$)
        \begin{solution}
            \[
            \lim_{y\to\infty}e^{x+iy}= e^{x}\lim_{y\to\infty} e^{iy}
            \]
            This limit does not exists; both real and imaginary parts of $e^{iy}$ oscillate between $-1$ and $1$.
        \end{solution}

        \part $\displaystyle{\lim_{y\to-\infty}e^{x+iy}}$ (fixed $x$)
        \begin{solution}
            \[
            \lim_{y\to-\infty}e^{x+iy}= e^{x}\lim_{y\to-\infty} e^{iy}
            \]
            This limit does not exists; both real and imaginary parts of $e^{iy}$ oscillate between $-1$ and $1$.
        \end{solution}
        \part $\displaystyle{\lim_{|z|\to\infty}e^z}$
        \begin{solution}
            This limit doesn't exist as $\lim_{|x|\to\infty}e^x$ doesn't.
        \end{solution}
        \part $\displaystyle{\lim_{|z|\to\infty}|e^z|}$
        \begin{solution}
            This limit doesn't exist as $\lim_{|x|\to\infty}|e^x|=\lim_{|x|\to\infty}e^x$ doesn't.
        \end{solution}
    \end{parts}

    \question 
    \begin{parts}
    \part Prove that $|a^b| = |a|^b$ for $a\in \CC$ and $b\in\RR$.
    \begin{solution}
        \begin{align*}
            |a^b| &= |e^{b\log a}|\\
            &= |e^{b(\log|a| + i\arg a)}|\\
            &= e^{b\log|a|} &\text{because $b$ is real}\\
            &= |a|^b.
        \end{align*}
    \end{solution}

    \part Prove that, for a fixed branch of $\log$, $a^{b+c}=a^ba^c$.

    \part Prove that, for a fixed branch of $\log$, $(ab)^c=a^cb^c$ valid for all complex $a,b,c$ such that $\log(ab)=\log a + \log b$.
    \begin{solution}
        \begin{align*}
            (ab)^c &= e^{c\log(ab)}\\
            &= e^{b(\log a + \log b)}&\text{if $\log(ab)=\log a + \log b$}\\
            &= e^{b\log a + c\log b}\\
            &= e^{b\log a}e^{c\log b}\\
            &= a^ba^c
        \end{align*}
    \end{solution}
\end{parts}

    \question
    Determine the set on which the function is analytic and compute its derivative.
    \begin{parts}
        \part $\dfrac{1}{(z^3-1)(z^2+2)}$

        \begin{solution}
            The function is not differentiable on its domain,
            \[
                z\neq 1, e^{2i\pi/3}, e^{4\pi/3}, \sqrt2 i, -\sqrt2 i.
            \]
            For all other $z$,
            \[
                \frac d{dz}\dfrac{1}{(z^3-1)(z^2+2)} = -(z^3-1)^{-2}3z^2(z^2+2)^{-1} - (z^3-1)(z^2+2)^{-2}2z.
            \]
        \end{solution}

        \part $\dfrac{1}{z + z^{-1}}$

        \begin{solution}
            The function is differentiable on its domain:
            \[
                z\neq 0, i, -i.
            \]
            For all other $z$,
            \[
                \frac d{dz}\dfrac{1}{z + z^{-1}} = -(z+z^{-1})^{-2}(1-z^{-2}).
            \]
        \end{solution}
        
        \part $\dfrac z{z^n-2}$

        \begin{solution}
        The function is differentiable on its domain:
        \[
            z\neq \sqrt2e^{2\pi ik/n},\quad n=0,\ldots,n-1.
        \]
        For all other $z$,
        \[
            \frac d{dz} \dfrac z{z^n-2} = (z^n-2)^{-1} + z(-2)(z^n-2)^{-2}nz^{n-1}.
        \]
        \end{solution}
    \end{parts}

    \question
    Let
    \[
        f(z)=\begin{cases}
            z^5/|z|^4&\text{if $z\neq 0$},\\
            0&\text{if $z=0$}.
        \end{cases}
    \]
    \begin{parts}
        \part Show that
        \[
            \lim_{z\to 0}\frac{f(z)}z
        \]
        does not exist.

        \begin{solution}
            \[
            \lim_{z\to 0}\frac{f(z)}z = \lim_{z\to 0} \frac{z^4}{|z|^4} = \lim_{z\to 0}\frac{z^4}{(z\bar z)^2}
            = \lim_{z\to 0}\frac{z^2}{\bar z^2}
            \]
            If $z\to 0$ along the real axis, $\bar z = z$ and the limit is $1$.
            
            Consider approaching $0$ along the line $z=re^{i\pi/4}$. Then
            \[
                \lim_{r\to 0}\frac{z^2}{{\bar z}^2} = \lim_{r\to 0} \frac{r^2e^{i\pi/2}}{r^2e^{-i\pi/2}}
                = \frac{i}{-i}=-1.
            \]
            Thus the limit does not exist.
        \end{solution}

        \part Let $u=\Re f$, $v=\Im f$. Show that
        \[
            u(x,0)=x,\quad u(0,y)=0,\quad v(x, 0)=0,\quad v(0, y)=y.
        \]

        \begin{solution}
            We have
            \[
                f(x + 0i) = \frac{x^5}{|x|^4} = \frac{x(x^4)}{x^4} = x + 0i
            \]
            and
            \[
                f(0 + iy) = \frac{{iy}^5}{|iy|^4} = \frac{iy(y^4)}{y^4} = 0 + iy.
            \]
            Therefore,
            \[
                u(x,0) = \Re f(x+0i) = x,\quad v(x,0) = \Im f(x+0i) = 0
            \]
            and
            \[
                u(0,y) = \Re f(0+iy) = 0,\quad v(0,y) = \Im f(0+iy) =y.
            \]
        \end{solution}

    \part Conclude that the partial derivatives of $u$ and $v$ with respect to $x$ and $y$ exist, that
    the Cauchy-Riemann equations are satisfied, but $f'(0)$ does not exist.
    Why does this not contradict the Cauchy-Riemann theorem? 

    \begin{solution}
        By definition, $f(0)=0$, i.e., $u(0,0)=v(0,0)=0$.
        Therefore,
        \begin{align*}
            u_x(0,0) &= \lim_{h\to 0}\frac{u(0+h,0) - u(0,0)}{h} = 1\\
            u_y(0,0) &= \lim_{h\to 0}\frac{u(0,0+h) - u(0,0)}{h} = 0\\
            v_x(0,0) &= \lim_{h\to 0}\frac{v(0+h,0) - v(0,0)}{h} = 0\\
            v_y(0,0) &= \lim_{h\to 0}\frac{v(0, 0+h)) - v(0,0)}{h} = 1\\
        \end{align*}
        Thus, the Cauchy-Riemann equations are satisfied.

        This doesn't contradict the Cauchy-Riemann theorem as aren't \emph{continuous} at $z=0$, a condition required by the theorem.
    \end{solution}
    \end{parts}

    \question
    Find the real and imaginary parts of the function and verify that they satisfy the Cauchy-Riemann equations.
    \begin{parts}
        \part $f(z)=z^3$

        \begin{solution}
            \[
                z^3 = (x+iy)^3 = x^3 + 3x^2iy + 3x(iy)^2 + (iy)^3 = (x^3 - 3xy^2) + i(3x^2y - y^3)
            \]
            \begin{align*}
                u_x &= 3x^2 -3y^2\\
                u_y &= -6xy\\
                v_x &= 6xy\\
                v_y &= 3x^2y - 3y^2.
            \end{align*}
            Thus, the Cauchy-Riemann equations are satisfied.
        \end{solution}

        \part $ze^{-z}$
        \begin{solution}
            \[
                ze^{-z} = (x+iy)e^{-x}(\cos y -i\sin y) = e^{-x}(x\cos y+y\sin y) + ie^{-x}(y\cos y - x\sin y)
            \]
            \begin{align*}
                u_x &= -e^{-x}(x\cos y + y\sin y) + e^{-x}\cos y\\
                &= e^{-x}(\cos y - x\cos y - y\sin y)\\
                u_y &= e^{-x}(-x\sin y + \sin y + y\cos y)\\
                v_x &= -e^{-x}(y\cos y - x\sin y) + e^{-x}\sin y\\
                &= e^{-x}(-y\cos y + x\sin y - \sin y)\\
                v_y &= e^{-x}(\cos y - y\sin y - x\cos y).
            \end{align*}
            Thus, the Cauchy-Riemann equations hold.
        \end{solution}

        \part $\cos 2z$
        
        \begin{solution}
            We have:
            \[
                \cos(2z) = \cos(2x)\cosh(2y) - i\sin(2x)\sinh(2y)
            \]
            \begin{align*}
                u_x &= -2\sin(2x)\cosh(2y)\\
                u_y &= 2\cos(2x)\sinh(2y)\\
                v_x &= -2\cos(2x)\sinh(2y)\\
                v_y &= -2\sin(2x)\cosh(2y)
            \end{align*}
            Thus, the Cauchy-Riemann equations hold.
        \end{solution}
    \end{parts}

    % \question
    % Let
    % \[
    %     u(x,y) = e^{-x}(x\sin y - y\cos y).
    % \]
    % Find a function $v(x,y)$ such that $f=u+iv$ is analytic.

    % \begin{solution}
    %     We compute:
    %     \begin{align*}
    %         u_x(x,y) &= e^{-x}\sin y - e^{-x}x\sin y + ye^{-x}\cos y\\
    %         u_y(x,y) &= xe^{-x}\cos y  - e^{-x}\cos y + ye^{-x}\sin y
    %     \end{align*}
    %     Since we want
    %     \begin{align*}
    %         v_y(x,y) &= u_x(x,y)\\
    %         &= -e^{-x}x\sin y -e^{-x}y\cos y + e^{-x}\sin y
    %     \end{align*}
    %     to hold, we let $v$ be an antiderivative of $u_x$ with respect to $y$:
    %     \begin{align*}
    %         v(x,y) &= \int (e^{-x}\sin y - e^{-x}x\sin y + ye^{-x}\cos y)\,dy\\
    %         &= xe^{-x}\cos y + ye^{-x}\sin y + C(x)
    %     \end{align*}
    %     We have the flexibility to choose $C(x)$ so that $u_y = -v_x$ holds.
    %     \[
    %         v_x(x,y) = e^{-x}\cos y -xe^{-x}\cos y - ye^x\sin y + C'(x)
    %     \]
    %     The identity $u_y=-v_x$ holds with $C=0$, so we take
    %     \[
    %         v(x,y) = e^{-x}(x\cos y + y\sin y).
    %     \]
    % \end{solution}

    % \question
    % Find an analytic function $f(z)$ such that $\Re f'(z)=3x^2 - 4y - 3y^2$ and $f(1+i)=0$.

    % \begin{solution}
    %     Write $f=ui+v$.
    %     Then $f'= u_x + iv_x$ and
    %     \[
    %         u_x = \Re f'(z) = 3x^2 - 4y - 3y^2.
    %     \]
    %     Thus,
    %     \[
    %         u = \int u_x\,dx = x^3 -4xy -3xy^2 + C(y).
    %     \]
    %     Since we want $f$ to be analytic, we need the Cauchy-Riemann equations to hold:
    %     \[
    %         v_x = -u_y = 4x + 6xy - C'(y).
    %     \]
    %     Thus,
    %     \[
    %         v = \int v_x\,dx = 2x^2 + 3x^2y - xC'(y) + D(y).
    %     \]
    %     We ensure that $u_x=v_y$ holds:
    %     \[
    %         v_y = 3x^2 - xC''(y) + D'(y)
    %     \]
    %     Thus, we need $D'(y)=-4y-3y^2$ and $C''(y)=0$.
        

        
    % \end{solution}

    % \question
    % Prove that $f(z)=|z|^4$ is differentiable but not analytic at $z=0$.
\end{questions}
\end{document}