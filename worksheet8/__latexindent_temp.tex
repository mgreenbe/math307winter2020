\documentclass[12pt]{exam}
\usepackage{fullpage, amsmath, amssymb, amsthm}
\usepackage{tikz} 
\usepackage[inline]{enumitem}


\renewcommand{\partshook}{\setlength\itemsep{1em}}
\newcommand{\D}{\displaystyle}
\newcommand{\CC}{\mathbb{C}}
\newcommand{\RR}{\mathbb{R}}
\newcommand{\dd}{\mathrm{d}}
\newcommand{\dz}{\mathrm{d}z}
\newcommand{\dx}{\mathrm{d}x}
\newcommand{\ZZ}{\mathbb{Z}}

\newcommand{\cic}{\frac1{2\pi i}\int_C}
\renewcommand{\epsilon}{\varepsilon}
\renewcommand{\Re}{\operatorname{Re}}
\renewcommand{\Im}{\operatorname{Im}}
\DeclareMathOperator{\Log}{Log}
\DeclareMathOperator{\Arg}{Arg}
\DeclareMathOperator{\sign}{sign}
\DeclareMathOperator{\length}{length}
\DeclareMathOperator*{\Res}{Res}
\DeclareMathOperator{\ord}{ord}

\begin{document}

\section*{MATH 307 --- Worksheet \#8 }

\bigskip
\begin{questions}
\setlength\itemsep{1em}
\setlength\parskip{1em}

\question
Evaluate the integral. All paths are positively oriented.
\begin{parts}
\part $\displaystyle \int_{|z|=1}\frac z{z^2+2z+5}\,dz$

\begin{solution}
We have $z^2 + 2z + 5 = (z-\alpha)(z-\bar\alpha)$, where $\alpha = -1 + 2i$.
Neither $\alpha$ nor $\bar\alpha$ are lie inside the unit circle. Therefore,
$$\int_{|z|=1}\frac z{z^2+2z+5}\,dz=0,$$
by Cauchy's Theorem.
\end{solution}

\part $\displaystyle \int_{|z|=9}\frac 1{e^z-1}\,dz$
    
\begin{solution}
The function $f(z)=1/(e^z-1)$ has a simple pole at $z=0$ with residue
\[
    \lim_{z\to 0} zf(z) = \lim_{z\to 0} \frac z{e^z-1} = 1,
\]
by l'Hopital's rule. It has no other singularities. Therefore, by the residue theorem,
\[
    \int_{|z|=9}\frac 1{e^z-1}\,dz = 2\pi i\cdot 1 = 2\pi i
\]
\end{solution}

\part $\displaystyle \int_{|z|=8}\tan z\,dz$

\begin{solution}
    The function $\tan z$ has singularities at the zeros of $\cos z$:
    \[
        z_k = \frac\pi 2+k\pi,\quad k\in\ZZ.
    \]
    These zeros are all simple. (Why?)
    Therefore, the $z_k$ are all simple poles of $\tan z$.
    Since $\tan z$ is $\pi$-periodic, these poles all have the same residue:
    \[
        \lim_{z\to\pi/2}z\tan z = \lim_{z\to\pi/2}\frac{z\sin z}{\cos z}
        = \lim_{z\to\pi/2}\frac{\sin z + z\cos z}{\sin z} = 1,
    \]
    by l'Hopital's rule.

    There are six singularities of $\tan z$ in the circle $|z|=8$:
    \[z_k=\pi/2 + k\pi,\quad k=-3, -2, -1, 0, 1, 2.\]
    Therefore, by the Residue Theorem,
    \[
        \int_{|z|=8}\tan z\,dz = 6\cdot 2\pi i\cdot 1 = 12\pi i.
    \]
\end{solution}

\part $\displaystyle \int_{|z|=3}\frac {5z-2}{z(z-1)}\,dz$
    
\begin{solution}
The function $f(z)=(5z-2)/z(z-1)$ has a simple poles at $z=0$ and at $z=1$ with residues
\[
    \lim_{z\to 0} zf(z) = \frac{5(0) - 2}{0 - 1} = 2
\]
and
\[
    \lim_{z\to 0} zf(z) = \frac{5(1) - 2}{1} = 3,
\]
respectively. Both of these singularities lie inside the circle $|z|=3$.
Therefore, by the residue theorem,
\[
    \int_{|z|=3}\frac {5z-2}{z(z-1)}\,dz = 2\pi i(2 + 3) = 10\pi i.
\]
\end{solution}

\part $\displaystyle \int_\gamma\frac {e^{-z^2}}{z^2}\,dz$, 
where $\gamma$ is the square with vertices $\pm 1\pm i$.
    
\begin{solution}
Since $e^{-z^2}\neq 0$ for all $z\in\CC$, $z=0$ is the only pole of $f(z)=e^{-z}/^2$; it has order $2$.
The Laurent expansion of $f(z)$ around $z=0$ contains only even powers of $z$. Therefore,
\[
    \Res_{z=0}f(z) = 0
\]
and
\[
    \int_\gamma\frac {e^{-z^2}}{z^2}\,dz=0
\]
by the Residue Theorem.
\end{solution}

\part $\displaystyle \int_{|z|=1/2}\frac1{(1-z)^3}\,dz$

\begin{solution}
    The function $f(z)=1/(1-z)^3$ is analytic on and inside $|z|=1/2$. Therefore,
    \[
        \int_{|z|=1/2}\frac1{(1-z)^3}\,dz = 0,
    \]
    by Cauchy's Theorem.
\end{solution}

\part $\displaystyle \int_{|z-1|=1/2}\frac1{(1-z)^3}\,dz$

\begin{solution}
    The function $f(z)=1/(1-z)^3$ has a single triple pole inside $|z-1|=1/2$, at $z=1$.
    Since the Laurent expansion of $f(z)$ around $z=1$ is
    \[
        f(z) = \frac{-1}{(z-1)^3}, 
    \]
    the residue at $z=1$ is $0$. Therefore,
    \[
        \int_{|z-1|=1/2}\frac1{(1-z)^3}\,dz = 0
    \]
    by the Residue Theorem.
\end{solution}

\part $\displaystyle \int_{|z-1|=1/2}\frac{e^z}{(1-z)^3}\,dz$

\begin{solution}
    The function $f(z)={e^z}/(1-z)^3$ has a single triple pole inside $|z-1|=1/2$, at $z=1$.
    The Laurent expansion of $f(z)$ around $z=1$ has the form
    \begin{align*}
        f(z) &= \frac{-1}{(z-1)^3}\left(e + e(z-1) + \frac e2(z-1)^2 + \cdots\right)\\
        &= \frac{-e}{(z-1)^3} + \frac{-e}{(z-1)^2} + \frac{-e/2}{z-1} + \cdots.
    \end{align*}
    Therefore, by the Residue Theorem,
    \[
        \int_{|z-1|=1/2}\frac{e^z}{(1-z)^3}\,dz = 2\pi i(-e/2) = -e\pi i.
    \]
\end{solution}

\part $\displaystyle \int_{|z|=3}\frac{\cos(z+2)}{z(z+2)^3}\,dz$

\begin{solution}
    The function $f(z)=\cos (z+2)/z(z+2)^3$ has a simple pole at $z=0$,
    a triple pole at $z=-2$, and no other singularities.
    \[
        \Res_{z=0}f(z) = \lim_{z\to 0}zf(z) = \frac{\cos 2}{(0+2)^3} = \frac{\cos 2}8.
    \]
    To compute the residue at $z=-2$, let $g(z)=\cos (z+2)/z$.
    Since $g(z)$ is analytic at $z=-2$, it's Laurent expansion around $z=-2$ is  a Taylor expansion:
    \[
        g(z) = a_0 + a_1(z+2) + a_2(z+2)^2 + \cdots.
    \]
    Therefore, the Laurent expansion of $f(z)$ around $z=-2$ has the form
    \[
        f(z) = \frac{a_0}{(z+2)^3} + \frac{a_1}{(z+2)^2} + \frac{a_2}{(z+2)^2} + \cdots.
    \]
    It follows that $\Res_{z=-2}f(z)=a_2$.
    We can compute $a_2$ by using the usual formula for the Taylor coefficient.
    \begin{align*}
        g'(z) &= \frac{-z\sin (z+2) + \cos (z+2)}{z^2},\\
        g''(z) &= \frac{(-(\sin (z+2) + z\cos (z+2)) - \sin (z+2))z^2 - 2z(-z\sin (z+2) + \cos (z+2))}{z^4}\\
        g''(-2) &= \frac{(-(-2)(1))(-2)^2 - 2(-2)(1)}{(-2)^4}\\
        &= \frac34\\
        a_2 &= \frac{g''(-2)}{2!}\\&= \frac38
    \end{align*}

    Therefore,
    \[
        \int_{|z|=3}\frac{\cos(z+2)}{z(z+2)^3}\,dz = 2\pi i\left(\frac{\cos 2}8 + \frac38\right) = \frac{\pi i}4(\cos 2 + 3).
    \]
\end{solution}
    
\end{parts}

\end{questions}
\end{document}