\documentclass[12pt]{exam}
\usepackage{fullpage, amsmath, amssymb, amsthm}
\usepackage{tikz} 
\usepackage[inline]{enumitem}


\renewcommand{\partshook}{\setlength\itemsep{1em}}
\newcommand{\D}{\displaystyle}
\newcommand{\CC}{\mathbb{C}}
\newcommand{\RR}{\mathbb{R}}
\newcommand{\dd}{\mathrm{d}}
\newcommand{\dz}{\mathrm{d}z}
\newcommand{\dx}{\mathrm{d}x}
\newcommand{\ZZ}{\mathbb{Z}}

\newcommand{\cic}{\frac1{2\pi i}\int_C}
\renewcommand{\epsilon}{\varepsilon}
\renewcommand{\Re}{\operatorname{Re}}
\renewcommand{\Im}{\operatorname{Im}}
\DeclareMathOperator{\Log}{Log}
\DeclareMathOperator{\Arg}{Arg}
\DeclareMathOperator{\sign}{sign}
\DeclareMathOperator{\length}{length}
\DeclareMathOperator{\res}{res}
\DeclareMathOperator{\ord}{ord}

\begin{document}

\section*{MATH 307 --- Worksheet \#7 }

\bigskip
\begin{questions}
    \setlength\itemsep{1em}
    \setlength\parskip{1em}

    \question
    Find the Laurent expansion of
    $$
    f(z)=\frac1{z(z^2+1)}
    $$
    valid in the given region.
    \begin{parts}
        \part $0<|z|<1$

        \begin{solution}
            \begin{align*}
                f(z) &= \frac1z\frac1{1-(-z^2)}\\
                &= \frac1z\sum_{n=0}^\infty (-z^2)^n\\
                &= \sum_{n=0}^\infty (-1)^n z^{2n-1}
            \end{align*}
        \end{solution}

        \part $|z|>1$

        \begin{solution}
            \begin{align*}
                f(z) &= \frac1{z^3}\frac1{1 - (-z^{-2})}\\
                &= \frac1{z^3}\sum_{n=0}^\infty (-z^{-2})^n\\
                &= \sum_{n=0}^\infty (-1)^n z^{-2n-3}
            \end{align*}
        \end{solution}
    \end{parts}

    \question
    Find the Laurent expansions of
    $$
    f(z)=\frac1{1+z^2} + \frac1{3-z}
    $$
    valid in the given region.

    \begin{parts}
        \part $|z|<1$
    
        \begin{solution}
            \begin{align*}
                f(z) &= \frac1{1-(-z^2)} + \frac13\frac1{1 - z/3}\\
                &= \sum_{n=0}^\infty (-z^2)^n + \frac13\sum_{n=0}^\infty \left(\frac z3\right)^n\\
                &= \sum_{n=0}^\infty (-1)^n z^{2n} + \sum_{n=0}^\infty \frac {z^n}{3^{n+1}}\\
            \end{align*}
        \end{solution}

        \part $1<|z|<3$

        \begin{solution}
            \begin{align*}
                f(z) &= \frac1{z^2}\frac1{1-(-z^{-2})} + \frac13\frac1{1 - z/3}\\
                &= \sum_{n=0}^\infty (-z^{-2})^n + \frac13\sum_{n=0}^\infty \left(\frac z3\right)^n\\
                &= \sum_{n=0}^\infty (-1)^n z^{-2n} + \sum_{n=0}^\infty \frac {z^n}{3^{n+1}}\\
            \end{align*}
        \end{solution}

        \part $|z|>3$

        \begin{solution}
            \begin{align*}
                f(z) &= \frac1{z^2}\frac1{1-(-z^{-2})} - \frac1z\frac1{1 - 3/z}\\
                &= \sum_{n=0}^\infty (-z^{-2})^n + \frac1z\sum_{n=0}^\infty \left(\frac 3z\right)^n\\
                &= \sum_{n=0}^\infty (-1)^n z^{-2n} + \sum_{n=0}^\infty \frac {3^n}{z^{n+1}}\\
            \end{align*}
        \end{solution}
    \end{parts}

    \question
    Find the poles of the $f(z)$. For each such pole, $a$, determine:
    \begin{itemize}
        \item $\ord_af(z)$,
        \item $\res_af(z)$,
        \item the annuli of convergence of the Laurent expansions of $f(z)$ around $a$.
    \end{itemize}

    \begin{parts}
        \part $\displaystyle f(z) = \frac{e^z(z-3)}{(z-1)(z-5)}$

        \begin{solution}
            The poles of $f(z)$, both simple, are at $1$ and $5$.

            \begin{align*}
                \res_1 f(z) &= \lim_{z\to 1}(z-1)\frac{e^z(z-3)}{(z-1)(z-5)} = \frac{e^1(1-3)}{1-5}=\frac e2\\
                \res_5 f(z) &= \lim_{z\to 5}(z-5)\frac{e^z(z-3)}{(z-1)(z-5)} = \frac{e^5(5-3)}{5-1}=\frac {e^5}2
            \end{align*}

            The annuli of convergence of the Laurent expansions of $f(z)$ around $z=1$ are $0 < |z-1| < 4$ and $|z-1| > 4$.

            The annuli of convergence of the Laurent expansions of $f(z)$ around $z=5$ are $0 < |z-5| < 4$ and $|z-5| > 4$.
        \end{solution}

        \part $\displaystyle f(z) = \frac{e^z-1}{z}$

        \begin{solution}
            Since $\ord_0(e^z-1)=1$ and $\ord_0z=1$, $f(z)$ has a pole of order $1-1=0$ at $z=0$.
            In other words, $z=0$ is a removable singularity of $f(z)$; $f(z)$ has no other singularities.
            
            Since $0$ is a removable singularity of $f(z)$, $\res_0 f(z)=0$.

            The annulus of convergence of the Laurent expansion of $f(z)$ around $z=0$ is $0<|z|<\infty$.
        \end{solution}

        \part $\displaystyle f(z) = \frac{e^z-2}{z}$

        \begin{solution}
            Since $e^z-2$ and $z$ vanish to orders $0$ and $1$, respectively, at $z=0$, $f(z)$ has a simple pole there.
            It has no other singularities.

            \begin{align*}
                \res_0 f(z) &= \lim_{z\to 0}z\frac{e^z-2}{z} = e^0-2=-1.
            \end{align*}

            The annulus of convergence of the Laurent expansion of $f(z)$ around $z=0$ is $0<|z|<\infty$.
        \end{solution}

        \part $\displaystyle f(z) = \frac{\cos z}{1-z}$

        \begin{solution}
            Since $\cos z$ and $1-z$ vanish to orders $0$ and $1$, respectively, at $z=1$, $f(z)$ has a simple pole there.
            It has no other singularities.

            \begin{align*}
                \res_1 f(z) &= \lim_{z\to 1}(z-1)\frac{\cos z}{1-z} = -\cos 1
            \end{align*}

            The annulus of convergence of the Laurent expansion of $f(z)$ around $z=1$ is $0<|z-1|<\infty$.
        \end{solution}

        \part $\displaystyle f(z)=\frac{z^2-1}{\cos(\pi z) + 1}$

        \begin{solution}
            Write $g(z)$ and $h(z)$ for the numerator and denominator of $f(z)$, respectively.
            $g(z)$ has simple zeros at $z=\pm 1$ while
            $h(z)$ has a zero when $\cos(\pi z) = -1$, i.e., when $z=k$, $k$ an odd integer.
            Since 
            \begin{multline*}
                h'(z) = -\pi\sin(\pi z),\quad h'(2k+1) = 0,\\
                h''(z) = -\pi^2\cos(\pi z),\quad h''(2k+1) = -\pi^2\neq 0,
            \end{multline*}
            $h(z)$ has a zero of order $2$ at each odd integer.
            It follows that
            $$
            \ord_1 f(z)=\ord_{-1} f(z) = 1,\quad \ord_k=-2,\quad\text{$k$ an odd integer $\neq\pm 1$}.
            $$

            Since $f(z)$ has simple poles at $z=\pm 1$,
            We have:
            $$
            \res_{\pm 1}f(z)=2\frac{g'(\pm 1)}{h''(\pm 1)} = 2\frac{\pm 2}{-\pi^2} = \mp\frac4{\pi^2}
            $$

            Let $k$ be an odd integer, $k\neq\pm 1$. Then
            $$
            \res_k f(z) = 2\frac{g'(k)}{h''(k)}-\frac23\frac{g(k)h'''(k)}{h''(k)^2}
            = 2\frac{2k}{-\pi^2} - \frac23\frac{0}{\pi^4} = -\frac{4k}{\pi^2}.
            $$
            The annuli of convergence of $f(z)$ around $k$ are
            $$
            0<|z-k| < 2,\quad 2<|z-k|<4,\quad  4<|z-k|<6,\quad \ldots.
            $$
        \end{solution}
    \end{parts}

    \question
    Find and classify the singularities of $f(z)$ (removable, pole of order $n$, essential singularity).

    \begin{parts}
        \part $f(z)=\sin\dfrac1z$

        \begin{solution}
            The only singularity of $f(z)$ is at $z=0$.
            It's an essential singularity as the Laurent expansion of $f(z)$ around $z=0$,
            $$
            \sum_{n=0}^\infty \frac{(-1)^n}{(2n+1)!}\frac1{z^{2n+1}},
            $$
            has infinitely many terms with negative powers of $z$.
        \end{solution}

        \part $f(z)=\csc\dfrac1z$

    \begin{solution}
        $f(z)$ has singularities at $z=0$ and $z=1/k\pi$, $k\in\ZZ$, $k\neq 0$.

        Since $1/k\pi\to 0$ as $k\to\infty$, $0$ is not an isolated singularity of $f(z)$.
        Therefore, it must be an essential singularity.

        Suppose $k\neq 0$. Then $f(z)$ has a simple pole at $1/k\pi$:
        $$
        \ord_{k\pi}\sin z = 1 \Longrightarrow \ord_{1/k\pi}\sin \frac1z = 1
        \Longrightarrow \ord_{1/k\pi}\csc\frac1z = -1.
        $$
    \end{solution}
    \end{parts}
\end{questions}
\end{document}