\documentclass[answers, 12pt]{exam}

\usepackage{fullpage, amsmath, amssymb, amsthm}

\newcommand{\disp}{\displaystyle}

\newcommand{\CC}{\mathbb{C}}
\newcommand{\RR}{\mathbb{R}}
\renewcommand{\Re}{\operatorname{Re}}
\renewcommand{\Im}{\operatorname{Im}}
\DeclareMathOperator{\Log}{Log}
\DeclareMathOperator{\Arg}{Arg}
\DeclareMathOperator{\sign}{sign}
\begin{document}

\title{MATH 307 -- Supplementary problems}
\author{Matthew Greenberg}
\date{\today}
\maketitle

\section{MATH 307 --- Tutorial --- 01.24.2020 }
\begin{questions}
    \question Express the following in the form $x+iy$:
    \begin{parts}
        \part $\dfrac{2i-1}{5+6i}$
        \begin{solution}
            $\dfrac{2i-1}{5+6i} =\dfrac{(2i-1)(5-6i)}{5^2+6^2}=\dfrac1{61}(7+16i)$
        \end{solution}

        \part $(3-2i)^3$
        \begin{solution}
            $(3-2i)^3 = 3^3 - 3^2(2i) + 3(2i)^2 - (2i)^3 = -9-46i$
        \end{solution}
    \end{parts}

    \question Let $z=x+iy$. Express the following in the form $u(x,y) + iv(x,y)$.
    \begin{parts}
        \part $1-z^2$
    \begin{solution}
        \[
1 - z^2 = 1-(x+iy)^2 = 1 - (x^2 - y^2 + 2ixy) = (1-x^2+y^2) - 2ixy
        \]
    \end{solution}

    \part $\dfrac1{z^2}$
    \begin{solution}
        \[
            \frac1{z^2} = \frac{\overline{z^2}}{|z^2|^2} = 
            \frac{{\bar z}^2}{|z|^4} = \frac{(x-iy)^2}{(x^2+y^2)^2}=
            \frac{{\bar z}^2}{|z|^4} = \frac{x^2 - y^2}{(x^2+y^2)^2}
            + i\frac{-2xy}{(x^2+y^2)^2}
        \]
    \end{solution}

    \part $z^3$
        \begin{solution}
            \begin{multline*}
                \frac1{z^3} = \frac{\overline{z^3}}{|z^3|^2}=
                \frac{{\bar z}^3}{|z|^6} = \frac{(x-iy)^3}{(x^2+y^2)^3}=\\
                \frac{x^3 + 3x(-iy)^2 + 3x^2(-iy) + (-iy)^3}{(x^2+y^2)^3}=
                \frac{x^3 - 3xy^2}{(x^2+y^2)^3} + i\frac{-3xy^2  + y^3}{(x^2+y^2)^3}
                \end{multline*}
        \end{solution}
    \end{parts}

    \question
    Verify the identities $\Re(iz) = -\Im(z)$ and $\Im(iz)=\Re(z)$.
    \begin{solution}
        \[\Re(iz) = \Re(i(x+iy)) = \Re(-y + ix) = -y = -\Im z\]
        \[\Im(iz) = \Im(i(x+iy)) = \Im(-y + ix) = x = \Re z\]
    \end{solution}
    
    

    \question
    For which $z$ does the identity $\Re(z^2)=\Re(z)^2$ hold?
    \begin{solution}
        With $z=x+iy$, $\Re(z^2)=x^2-y^2$ and $\Re(z)^2=x^2$.
        Thus, $\Re(z^2)=\Re(z)^2$ holds if and only if $y=0$, i.e., if and only if $z\in\RR$.
    \end{solution}

    \question
    Express $\dfrac{i^3(1-i)}{2(1+i\sqrt3)}$ in the form $re^{i\theta}$ with $r>0$ and $\theta\in[5\pi, 7\pi)$.
    \begin{solution}
        We have
        \[
            r = \left|\dfrac{i^3(1-i)}{2(1+i\sqrt3)}\right| = \dfrac{|i^3||1-i|}{|2||1+i\sqrt3|}
            = \dfrac{1\sqrt{2}}{2\sqrt4} =\frac1{2\sqrt2}
        \]
        and
        \begin{align*}
            \arg\dfrac{i^3(1-i)}{2(1+i\sqrt3)}&= 3\arg i + \arg(1-i) - \arg 2 - \arg(1+i\sqrt3) + 2k\pi\\
            &=3\frac\pi 2 - \frac\pi 4 - 0 - \frac\pi 3 + 2k\pi\\
            &= \frac{11\pi}{12}+ 2k\pi.
        \end{align*}
        As
        \[
            11\pi/12+2k\pi\in [5\pi, 7\pi) \Longleftrightarrow k=3,
        \]
        we set
        \[
            \theta := \frac{11\pi}{12} + 6\pi = \frac{83\pi}{12}\in[5\pi, 7\pi).
        \]
        Thus,
        \[
            \dfrac{i^3(1-i)}{2(1+i\sqrt3)} = \frac1{2\sqrt2}e^{83\pi i/12}.
        \]
    \end{solution}

    \question
    Let $a,b,c,d\in\RR$ be such that $cd\neq 0$ and let $z\in\CC\setminus\RR$.
    \begin{parts}
    \part Express $\Im\dfrac{az+b}{cz+d}$ in terms of $\Im z$.
    \begin{solution}
        \[
            \frac{az+b}{cz+d} = \frac{(az+b)\overline{(cz+d)}}{|cz+d|^2}
            = \frac{ac|z|^2 + adz + bc\bar z + bd}{|cz+d|^2}
        \]
        Note that $ac|z|^2$, $bd$, and $|cz+d|^2$ are real with $|cz+d|^2>0$. Therefore,
        \begin{multline*}
            \Im \frac{ac|z|^2 + adz + bc\bar z + bd}{|cz+d|^2} = \Im \frac{adz + bc\bar z}{|cz+d|^2}\\
            =\frac{ad}{|cz+d|^2}\Im z + \frac{bc}{|cz+d|^2}\Im \bar z
            =\frac{ad-bc}{|cz+d|^2}\Im z.
        \end{multline*}
    \end{solution}

    \part When is $\Im\dfrac{az+b}{cz+d}$ equal to $0$?
    \begin{solution}
        As $z\in\CC\setminus\RR$, $\Im z\neq 0$. Therefore, $\Im\dfrac{az+b}{cz+d}=0$ if and only if $ad-bc=0$.
    \end{solution}
    \end{parts}


    \question
    Describe and sketch the set solution set.
    \begin{parts}
        \part $|z-i|=2$
        \begin{solution}
            The set of points at distance 2 from $i$, i.e., the circle centered at $i$ with radius $2$.
        \end{solution}
        \part $|z + i| = |z - 1|$
        \begin{solution}
            The set of points equidistant from $-i$ and $1$. These lie on the line $y=-x$.
        \end{solution}
        \part $|z+2i| + |z - 2i| = 6$
        \begin{solution}
            The set of points the sum of whose distances from $-2i$ and $2i$ is equal to $6$ is an ellipse with foci at $-2i$ and $2i$.
            Let's find its radii.
            By symmetry, the axes of the ellipse lie along the real and imaginary axes.
            Suppose $\pm ai$, $a>0$, are the points of the ellipse along the imaginary axis. Then
            \[
                6 = |-2i - ai| + |2i - ai| = (a+2) + (a-2) = 2a \Longrightarrow a=3
            \]
            Suppose $\pm bi$, $b>0$, are the points of the ellipse along the real axis. By the Pythagorean theorem,
            \[
                \left(\frac62\right)^2 = 2^2 + b^2 \Longrightarrow b=\sqrt5
            \]
            Thus, the Cartesian equation of the ellipse is
            \[
                \left(\frac {x}{\sqrt 5}\right)^2 + \left(\frac {y}{3}\right)^2 = 1.
            \]
        \end{solution}
        \part $|z+3| - |z-3| = 4$
        \begin{solution}
            The set of points the difference of whose distances from $(-3,0)$ and $(3,0)$ is equal to $4$ is an ellipse with foci at $-3i$ and $3i$.
    Let $(\pm a, 0)$, $a>0$, be the intersection points of this hyperbola with the $x$-axis.
    Since $(a,0)$ is on the curve,
    \[
    4 = |a+3| - |a-3| = a+3 - (3-a) = 2a \Longrightarrow a=2
    \]
    Let $b=\sqrt{3^2 - a^2}=\sqrt5$. Then the hyperbola has cartesian equation
    \[
                \left(\frac {x}{2}\right)^2 - \left(\frac {y}{\sqrt 5}\right)^2 = 1.
            \]
        \end{solution}
        
        \part $\Im z^2=4$
        \begin{solution}
            As $\Im z^2 = 2xy$,
            \[
                \Im z^2=0\Longleftrightarrow 2xy=0\Longleftrightarrow x=0\;\text{or}\;y=0.
            \]
            Thus, the solution set of $\Im z^2=0$ is the union of the real and imaginary axes.
        \end{solution}
    \end{parts}

    \question Solve the equation.
    
    \begin{parts}
        \part $z^2 + 2z + (1-i) = 0$
        \begin{solution}
            By the quadratic formula and a bit of algebra,
            \[
                z = \frac{-2\pm\sqrt{2^2-4(1)(1-i)}}{2} = -1\pm\sqrt i.
            \]
            The square roots of $i$ are $\pm\dfrac{1+i}{\sqrt2}$. Therefore,
            \[
                z = -1 \pm \dfrac{1+i}{\sqrt2}.
            \]
        \end{solution}
        \part $z^2 + (2i-3)z + 5 - i = 0$
    \begin{solution}
        By the quadratic formula a bit of algebra,
        \[
            z = \frac{(3-2i)\pm \sqrt{-15-8i}}2.
        \]
        To find the square roots of $-15-8i$, we convert to polar coordinates:
        \[
            r:= |-15-8i| = 17,\quad\quad \tan\theta = \frac8{15}
        \]
        (We \emph{cannot} write $\theta = \arctan (8/15)$. Why not?)
        Since $\pi/2<\theta/2<\pi$,
        \[
            \cos\frac\theta2 = -\sqrt{\frac{\cos\theta + 1}2}
            =-\sqrt{\frac{\frac{-15}{17} + 1}2} = -\frac1{\sqrt{17}}.
        \]
        (It's easy to miss this minus sign!)
        \[
            \sin\frac\theta2 = \sqrt{1-\cos^2\frac\theta2} = \sqrt{1-\frac1{17}} = \frac4{\sqrt{17}}
        \]
        Therefore, the square roots of $-15-8i$ are
        \[
            \pm \sqrt{r}\left(\cos\frac\theta2 + i\sin\frac\theta2\right)
            = \pm\sqrt{17}\left(-\frac1{\sqrt{17}} + \frac {4i}{\sqrt{17}}\right)
            = \pm(-1+ 4i)
        \]
        Therefore,
        \[
            z = \frac{(3-2i)\pm (-1+4i)}2 = 1+i,\; 2-3i.
        \]
    \end{solution}
\end{parts}

\question\label{Q:sqrt}
Given $x,y\in\RR$, show that 
\begin{equation}\label{E:sqrt}
    a=\sqrt{\frac{x+\sqrt{x^2+y^2}}2}\quad
    b=\sign(y)\sqrt{\frac{-x+\sqrt{x^2+y^2}}2}.
\end{equation}
is the unique solution to $(a+ib)^2=x+iy$ with $a\geq 0$.
\begin{solution}
    Suppose $(a+bi)^2 = x+iy$. Equating real and imaginary parts,
    \[
        (a+bi)^2 = x+iy\Longleftrightarrow (a^2-b^2) + 2iab = x + iy.
        \Longleftrightarrow a^2-b^2=x\;\text{and}\;2ab=y
    \]
    We consider the case $y\neq 0$, the case $y=0$ being simpler. If $y\neq 0$, then $a\neq 0$ and $b=y/2a$. Substituting this into $a^2-b^2=x$, we get
    \[
        a^2 - \frac{y^2}{4a^2} = x\Longleftrightarrow 4(a^2)^2 - 4x(a^2) - y^2=0
    \]
    \[
        a^2 = \frac{4x\pm \sqrt{16x^2 + 16y^2}}{8} = \frac{x\pm \sqrt{x^2 + y^2}}{2}
    \]
    Noting that $x-\sqrt{x^2+y^2}<0$,
    \[
        a^2=\frac{x-\sqrt{x^2+y^2}}2
    \]
    has no solution $a\in\RR$. Thus,
    \[
        a = \sqrt{\frac{x+\sqrt{x^2+y^2}}2}.
    \]
    We solve for $b$:
    \begin{multline*}
        \frac1{\sqrt2a} = \sqrt{\frac1{x+\sqrt{x^2+y^2}}} = \sqrt{\frac{x - \sqrt{x^2+y^2}}{x^2 - (x^2+y^2)}}\\ = \sqrt{\frac{-x+\sqrt{x^2+y^2}}{y^2}}
        =\sign(y)\frac{\sqrt{-x+\sqrt{x^2+y^2}}}{y}
    \end{multline*}
    Thus,
    \[
b = \frac{y}{2a} = \sign(y)\sqrt{\frac{-x+\sqrt{x^2+y^2}}{2}}.
    \]
\end{solution}


\question Prove the identities:
\[
    \cos z = \cosh(iz),\quad \cos(iz) = \cosh z,\quad
    \sin z = -i\sinh(iz),\quad \sin(iz) = i\sinh z
\]
\begin{solution}
    The identity $\cos z=\cosh(iz)$ follows directly from the definitions and
    \[
        \sin z = \frac{e^{iz} - e^{-iz}}{2i} = \frac1i\frac{e^{iz} - e^{-iz}}{2} = -i\sinh(iz)
    \]
    Replacing $z$ with $iz$ in these identities, we get
    \[
        \cos(iz) = \cosh(i(iz)) = \cosh(-z) = \cosh z,
    \]
    \[
        \sin(iz) = -i\sinh(i(iz)) = -i\sinh(-z) = i\sinh z.
    \]
\end{solution}

\question
Prove the identities:
\begin{align*}
    \cos(x+iy) &= \cos x\cosh y - i\sin x\sinh y,\\
    \sin(x+iy) &= \sin x\cosh y + i\cos x\sinh y
\end{align*}

\begin{solution}
    Using the identities proved above,
    \begin{align*}
        \cos(x+iy) &= \cos x\cos(iy) - \sin x\sin(iy)\\
        &=\cos x\cosh y - \sin x(i\sinh y)\\
        &=\cos x\cosh y - i\sin x\sinh y,
    \end{align*}
    \begin{align*}
        \sin(x+iy) &= \sin x\cos(iy) + \cos x\sin(iy)\\
        &=\sin x\cosh y + \cos x(i\sinh y)\\
        &=\sin x\cosh y + i\cos x\sinh y.
    \end{align*}
\end{solution}

\question
Prove the identity:
\[
    |\cos z|^2 = \cos^2 x + \sinh^2 y
\]
Deduce that
\[
    \lim_{y\to\infty} |\cos z| = \infty\quad\text{and}\quad
    \lim_{y\to\infty} |\sin z| = \infty
\]
\begin{solution}
    \begin{align*}
    |\cos z|^2 &= \cos^2 x\cosh^2 y + \sin^2 x\sinh^2 y\\
    &=  \cos^2 x\cosh^2 y + (1-\cos^2 x)\sinh^2 y\\
    &= \cos^2x(\cosh^2 y - \sinh^2 y) + \sinh^2 y\\
    &= \cos^2x + \sinh^2 y
\end{align*}

\begin{align*}
    |\sin z|^2 &= \sin^2 x\cosh^2 y + \cos^2 x\sinh^2 y\\
    &=  \sin^2 x\cosh^2 y + (1-\sin^2 x)\sinh^2 y\\
    &= \sin^2x(\cosh^2 y - \sinh^2 y) + \sinh^2 y\\
    &= \sin^2x + \sinh^2 y
\end{align*}

Since $\sinh y\to\infty$ as $y\to\infty$ and $|\sinh y|\leq |\cos z|,\;|\sin z|$ by the above ,
the statement about the limits follows.
\end{solution}


\question Solve the equation $|\cot z|=1$
    \begin{solution}
        We have:
        \begin{align*}
            &
            &|\cot z| &= 1\\
            \Longleftrightarrow&
            &|\cos z|^2 &= |\sin z|^2\\
            \Longleftrightarrow&
            &\cos^2 x+\sinh^2 y &= \sin^2 x+ \sinh^2 y\quad\text{(see above)}\\
            \Longleftrightarrow&
            &\cos^2 x &= \sin^2 x\\
            \Longleftrightarrow&
            &\cos x &= \pm \sin x\\
            \Longleftrightarrow&
            & x &= \pi \pm \frac\pi4\\
        \end{align*}
        Thus,
        \[
            |\cot z| = 1\Longleftrightarrow\Re z = k\pi \pm \frac\pi4.
        \]
    \end{solution}

    \question
Find all solutions:
\begin{parts}
    \item $e^{\bar{z}}=\overline{e^z}$
    \begin{solution}
        We have:
        \begin{multline*}
            e^{\bar z} = e^{x-iy} = e^x(\cos(-y)) + i\sin(-y))\\ = e^x(\cos y - i\sin y) = 
            \overline{e^x(\cos y + i\sin y)} = \overline{e^z}.
        \end{multline*}
        Thus, this identity holds for all $z$.
    \end{solution}

    \item $\cos(i\bar z) = \overline{\cos(iz)}$
    \begin{solution}
        We showed above that $e^{\bar z} = \overline{e^z}$. Therefore,
        \[
            \cos(i\bar z) = \cosh\bar z = \frac{e^{\bar z} + e^{-\bar z}}2
            = \overline{\frac{e^{z} + e^{- z}}2}
            = \overline{\cosh z}
            = \overline{\cos(iz)}
        \]
        for all $z$.
    \end{solution}

    
\end{parts}


\end{questions}


\section{The Complex Plane and Elementary Functions}
\begin{questions}
\question
Simplify:
\begin{parts}
    \part $(2-3i)(i+8)$
    \begin{solution}
        $(2-3i)(i+8)=19-22i$
    \end{solution}
    
    \part $(3-2i)^2$
    \begin{solution}
        $(3-2i)^2=5 -12i$
    \end{solution}
    
    \part $(1+i)^4$
    \begin{solution}
        $(1+i)^4 = 1^4 + 4i + 6i^2 + 4i^3 + i^4 =-4$
    \end{solution}
\end{parts}

\question
Find real and imaginary parts in terms of $x=\Re(z)$ and $y=\Im(z)$.
\begin{parts}
    
    \part $\dfrac z{z+1}$
    \begin{solution}
        \begin{multline*}
            \frac z{z+1} = \frac{x+iy}{(x+1) + iy} = \frac{(x+iy)((x+1) - iy)}{(x+1)^2 + y^2}\\ = \frac{x(x+1)+y^2 - ixy + i(x+1)y}{(x+1)^2 + y^2}
            =\frac{x(x+1)+y^2 +iy}{(x+1)^2 + y^2}
        \end{multline*}
        Therefore, $\Re\dfrac z{z+1}=\dfrac{x(x+1)+y^2}{(x+1)^2 + y^2}$
        and 
        $\Im\dfrac z{z+1}=\dfrac{y}{(x+1)^2 + y^2}$.
    \end{solution}
\end{parts}

\question
Is the identity $\Re(wz)=\Re(w)\Re(z)$ valid? Prove or give a counterexample.
\begin{solution}
    No, it isn't. For example,
    \[
        \Re(i\cdot i) = \Re(-1) = 1 \neq 1 = 1\cdot 1 = \Re(i)\Re(i).
    \]
\end{solution}



\question
Evaluate $\left|\dfrac{i(i+1)^3(4i+5)}{(2+3i)^2}\right|$.
\begin{solution}
    \[
        \left|\dfrac{i(i+1)^3(4i+5)}{(2+3i)^2}\right| = \dfrac{|i||i+1|^3|4i+5|}{|2+3i|^2}
        =\frac{1\cdot \sqrt{2}^3\sqrt{41}}{\sqrt{13}^2} = \frac{2\sqrt{82}}{13}
    \]
\end{solution}



\question $z(\bar z+2)=3$
        \begin{solution}
            The solutions of this equation are $z=-1$ and $z=3$:
            \begin{align*}
                z(\bar z+2)=3&\Longleftrightarrow
                |z|^2 + 2z - 3 = 0\\
                &\Longleftrightarrow (x^2+y^2+2x-3) + 2iy = 0\\
                &\Longleftrightarrow x^2+y^2+2x-3=0\;\text{and}\;y=0\\
                &\Longleftrightarrow x^2-2x-3=0\;\text{and}\;y=0\\
                &\Longleftrightarrow (x=-1\;\text{or}\;x=3)\;\text{and}\;y=0.
            \end{align*}
        \end{solution}



\question\label{Q:sqrts} Find the square roots of:
\begin{parts}
    \part\label{P:i} $i$
    \begin{solution}
        By~\eqref{E:sqrt} with $x=0$ and $y=1$, the square roots of $i$ are $\pm(a+ib)$, where
        \[
            a = \frac1{\sqrt2},\quad b=\frac1{\sqrt2}.
        \]
        Alternatively, since $i=e^{i\pi/2}$, the square roots of $i$ are
        \[
            \pm e^{i\pi/4} = \pm\left(\cos\frac\pi 4 + i\sin\frac\pi 4\right)
            = \pm\left(\frac1{\sqrt2} + \frac i{\sqrt2}\right).
        \]
    \end{solution}
    \part\label{P:-i} $-i$
    \begin{solution}
        By~\eqref{E:sqrt} with $x=0$ and $y=-1$, the square roots of $-i$ are $\pm(a+ib)$, where
        \[
            a = \frac1{\sqrt2},\quad b=-\frac1{\sqrt2}.
        \]
        Alternatively, since $-i=e^{-i\pi/2}$, the square roots of $i$ are
        \[
            \pm e^{-i\pi/4} = \pm\left(\cos\left(-\frac\pi 4\right) + i\sin\left(-\frac\pi 4\right)\right)= \pm\left(\frac1{\sqrt2} - \frac i{\sqrt2}\right).
        \]
        \[
            \cos\frac\pi 8
        \]
    \end{solution}
    \part\label{P:1+i} $1+i$
    \begin{solution}
        By~\eqref{E:sqrt} with $x=1$ and $y=1$, the square roots of $1+i$ are $\pm(a+ib)$, where
        \[
            a = \sqrt{\frac{1+\sqrt2}{2}},\quad b=\sqrt{\frac{-1+\sqrt2}{2}}.
            \tag{$*$}
        \]
        Alternatively, the square roots of $1+i=\sqrt2e^{i\pi/4}$ are
        \[
            \pm\sqrt[4]{2}e^{i\pi/8} = \pm\sqrt[4]{2}\left(\cos\frac\pi8 + i\sin\frac\pi8\right)
        \]
        We express $\cos(\pi/8)$ and $\sin(\pi/8)$ in terms of radicals.
        Using the identity $\cos 2x = 2\cos^2x-1$, 
        \[
            \cos\frac\pi4 = 2\cos^2\frac\pi8 - 1
        \]
        Solving for $\cos(\pi/8)$, we get
        \[
            \cos\frac\pi 8 = \sqrt{\frac{1+\cos\frac\pi4}2} = \sqrt{\frac{1+\frac1{\sqrt2}}2} 
            = \sqrt{\frac{\frac{\sqrt2+1}{\sqrt2}}2} = \sqrt{\frac{\frac{2+\sqrt2}{2}}2} =
            \frac{\sqrt{\sqrt2 + 2}} 2
        \]
        By the Pythagorean theorem,
        \[
            \sin^2\frac\pi 8 = 1 - \cos^2\frac\pi 8 = 1-\frac{2+\sqrt2}4 = \frac{2-\sqrt2}4.
        \]
        Therefore,
        \[
            \sin\frac\pi 8 = \frac{\sqrt{2-\sqrt2}}2
        \]
        and
        \[
            \pm\sqrt[4]{2}\left(\cos\frac\pi8 + i\sin\frac\pi8\right)= \pm2^{-3/4}\left(\sqrt{2+\sqrt2}+ i\sqrt{2-\sqrt2}\right).
            \tag{$**$}
        \]
        I leave it to you to reconcile $(*)$ and $(**)$.
    \end{solution}
\end{parts}

\question
Find all solutions:
\begin{parts}
    \part $z^2 + 4z -(4 - 6i) = 0$
    \begin{solution}
        By the quadratic formula and a bit of algebra,
        \[
        z = -2\pm\sqrt{8-6i}.
        \]
        To find the square roots of $8-6i$, we convert to polar coordinates:
        \[
            r:= |8-6i| = \sqrt{8^2 + 6^2} = 10,\quad \theta := \arctan\left(-\frac68\right)
        \]
        \[
            \cos\frac\theta2 = \sqrt{\frac{\cos\theta + 1}2}
            =\sqrt{\frac{\frac8{10} + 1}2} = \frac{3}{\sqrt{10}}
        \]
        \[
            \sin\frac\theta2 = \sqrt{1-\cos^2\frac\theta2} = \sqrt{1-\frac9{10}} = \frac1{\sqrt{10}}
        \]
        Therefore, the square roots of $8-6i$ are
        \[
            \pm \sqrt{r}\left(\cos\frac\theta2 + i\sin\frac\theta2\right) 
            = \pm\sqrt{10}\left(\frac3{\sqrt{10}} + \frac i{\sqrt{10}}\right)
            = \pm(3-i)
        \]
        Therefore,
        \[
            z = -4\pm(3-i) = 1 - i,\; -5+i.
        \]

    \end{solution}
    
    \part $z^4 - 1 = 0$
    \begin{solution}
        \[
            z^4-1 = (z^2-1)(z^2+1) = (z-1)(z+1)(z-i)(z+i)
        \]
        Therefore,
        \[
            z = -1,\;1,\;-i,\;i.
        \]
    \end{solution}
    \part $z^4 + 1 = 0$
    \begin{solution}
        \[
            z^4+1 = z^4-i^2 = (z^2-i)(z^2+i)
        \]
        By~\ref{Q:sqrts}(\ref{P:i}), the square roots of $i$ are $\pm(1+i)/\sqrt2$.
        By~\ref{Q:sqrts}(\ref{P:-i}), the square roots of $-i$ are $\pm(1-i)/\sqrt2$
        Thus, the solutions of $z^4+1=0$ are
        \[
            z=\frac{\pm1\pm i}{\sqrt2},
        \]
        with all four possible combinations of signs.
    \end{solution}
    \part $z^8 - 1 = 0$
    \begin{solution}
        \[
            z^8-1 = (z^4-1)(z^4+1)
        \]
        Therefore, by (b) and (c), the solutions of $z^8-1=0$ are
        \[
            z = \pm 1,\;\pm i,\;\frac{\pm1\pm i}{\sqrt2}\,(\text{all four sign combinations}).
        \]
    \end{solution}
    \part $x^4 - i = 0$
    \begin{solution}
        \[
            z^4-i = \left(z^2 - \frac1{\sqrt2}(1+i)\right)\left(z^2 + \frac1{\sqrt2}(1+i)\right)
        \]
        By~\ref{Q:sqrts}(\ref{P:1+i}), the square roots of $1+i$ are $\pm(a+ib)$, where
            \[
            a = \sqrt{\frac{1+\sqrt2}{2}},\quad b=\sqrt{\frac{-1+\sqrt2}{2}}.
        \]
        Therefore, the square roots of $(1+i)/\sqrt2$ are $\pm2^{-1/4}(a+ib)$.
        The square roots of $-(1+i)/\sqrt2$ are $i$ times the square roots of $(1+i)/\sqrt2$:
        $\pm2^{-1/4}(-b+ia)$. Thus, the solutions of $z^4-i=0$ are
        \[
            z = \pm2^{-1/4}(a+ib),\; \pm2^{-1/4}(-b+ia).
        \]
    \end{solution}
\end{parts}

\question
Suppose $\Re z>0$ and $\Re w>0$. Show that
\[
    \Log(wz) = \Log w + \Log z.
\]
\begin{solution}
    Since $\Re z>0$, $-\pi/2<\Arg z< \pi/2$. Thus, $-\pi < \Arg z + \Arg w< \pi$. It follows that
    $\Arg(wz) = \Arg w + \Arg z$. Therefore,
    \begin{multline*}
        \Log(wz) = \log |wz| + i \Arg(wz) = \log|w| + \log |z| + i(\Arg w + \Arg z)\\
        = (\log|w| + i\Arg w )+ (\log |z| + i\Arg z) = \Log w + \Log z.
    \end{multline*}
\end{solution}







\question Write all values of the following expressions in the form $x+iy$.
\begin{parts}
    \item $\Log(\Log i)$
    \begin{solution}
        \[
            \Log i = \Log |i| + i\Arg i = \log 1 + \frac{i\pi}2
            = \frac{i\pi}2
        \]
        \[
            \log(\Log i) = \Log\frac{i\pi}2 = \log\left|\frac{i\pi}2\right| + i\arg\frac{i\pi}2
            = \log\frac\pi 2 + i\left(\frac{\pi}2+2k\pi\right)
        \]
    \end{solution}

    \item $\sin(e^i)$
    \begin{solution}
        \begin{align*}
            \sin(e^i) &= \sin(\cos 1 + i\sin 1)\\ &= \sin(\cos 1)\cosh(\sin 1) + i\cos(\cos 1)\sinh(\sin 1)
        \end{align*}
    \end{solution}

    \item $(-3)^{\sqrt2}$
    \begin{solution}
        \begin{align*}
            (-3)^{\sqrt 2} &= e^{\sqrt2\log(-3)} = e^{\sqrt2(\log|-3| + i\arg(-3))}\\
            &=e^{\sqrt2(\log3 + i(\pi + 2k\pi))}\\
            &= e^{\sqrt2\log 3}e^{i(\pi + 2k\pi)}\\
            &= 3^{\sqrt2}\left(\cos\sqrt2(\pi + 2k\pi) + i\sin\sqrt2(\pi + 2k\pi)\right)
        \end{align*}
    \end{solution}
\end{parts}

\question
Find all solutions:
\begin{parts}
    \item $e^{\bar{z}}=\overline{e^z}$
    \begin{solution}
        We have:
        \begin{multline*}
            e^{\bar z} = e^{x-iy} = e^x(\cos(-y)) + i\sin(-y))\\ = e^x(\cos y - i\sin y) = 
            \overline{e^x(\cos y + i\sin y)} = \overline{e^z}.
        \end{multline*}
        Thus, this identity holds for all $z$.
    \end{solution}

    \item $\sinh z + \cosh z = i$
    \begin{solution}
        We have:
    \[          
        \sinh z + \cosh z = \frac{e^z - e^{-z}}{2} + \frac{e^z + e^{-z}}{2} = e^z
    \]
    $e^z = i$ if and only if $z=i(\pi/2+2k\pi)$.
    \end{solution}
\end{parts}
 

    

\question
Find the argument of $\dfrac{3+3i}{\sqrt3 + i}$ in the interval $[5\pi, 7\pi)$.

\question
Using the principal branch of the logarithm, compute:
\begin{parts}
    \part $\Log(1 + i\sqrt3)$
    \begin{solution}
        \[
            \Log(1+i\sqrt3) = \log|1+i\sqrt3| + i\Arg(1+i\sqrt 3) = \log 2 + i\frac\pi 3
        \]
    \end{solution}
    \part $(1+i)^{1+i}$
    \begin{solution}
        \begin{align*}
           i^{1+i} &= e^{i\Log(1+i)}\\
           &= e^{i(\log|1+i| + i\Arg(1+i))}\\
           &= e^{i(\log 2 + i\pi/4)}\\
           &= e^{-\pi/4 + i\log 2}\\
           &=e^{-\pi/4}(\cos(\log 2) + i\sin(\log 2))\\
        \end{align*}
    \end{solution}
\end{parts}

\question
Compute $|e^{i\pi^2}|$.

\begin{enumerate}
    \item $e^{\Log z} = z$ for all $z$
    \item $\Log e^z = z$ for all $z$.
\end{enumerate}

Which of the following statements is true?
\begin{enumerate}
    \item Only (1) is correct.
    \item Only (2) is correct.
    \item Both (1) and (2) are correct.
    \item Neither (1) nor (2) are correct
\end{enumerate}

Choose a branch of $\log z$ such that $\log z$ is continuous on the positive real axis and
\[
e^{\frac12\log x} = -\sqrt x
\]
for all $x>0$. Justify your answer.

Choose a branch of the square root function such that
    $\sqrt 1=1$ and $\sqrt i = -\dfrac{1+i}{\sqrt 2}$.
    Justify your answer.

Solve $e^{2z} = -1+\sqrt 3$

Solve: $|e^{iz}|=2$.
What's wrong with the following argument?
\[
    |e^{iz}|^2= |\cos z + i\sin z|^2 = \cos^2 z + \sin^2 z = 1 \quad\text{for all $z\in\CC$.}
\]

\question
\begin{parts}
\part Does the identity $\overline{\Log z}=\Log\bar z$ hold for all $z$ in the domain of continuity of $\Log z$? Prove or give a counterexample.
\begin{solution}
    Noting that $\Arg\bar z = -\Arg z$ for all $z$ in the domain of continuity of $\Log z$,
    \[
        \overline{\Log z}=\overline{\log|z| + i\Arg z} = \log|z| -i\Arg z = \log|\bar z| + i\Arg\bar z = \Log\bar z
    \]
\end{solution}
\part 
Let $\log z$ denote the branch of the logarithm for which $\arg z\in [0,2\pi)$.
    Does the identity $\overline{\Log z}=\Log\bar z$ hold for all $z$ in the domain of continuity of $\Log z$? Prove or give a counterexample.
    \begin{solution}
        The identity does not hold for $z=i$:
        \begin{align*}
            \log \bar i &= \log(-i)=\log|-i| + i\arg(-i) = i\frac{3\pi}2,\\
            \overline{\log i} &= \overline{\log|i| + i\arg i} = \overline{i\frac\pi2} = -i\frac\pi2
        \end{align*}
    \end{solution}
\end{parts}




\end{questions}

\end{document}