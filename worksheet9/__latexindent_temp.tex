\documentclass[12pt]{exam}
\usepackage{fullpage, amsmath, amssymb, amsthm}
\usepackage{tikz} 
\usepackage[inline]{enumitem}


\renewcommand{\partshook}{\setlength\itemsep{1em}}
\newcommand{\D}{\displaystyle}
\newcommand{\CC}{\mathbb{C}}
\newcommand{\RR}{\mathbb{R}}
\newcommand{\dd}{\mathrm{d}}
\newcommand{\dz}{\mathrm{d}z}
\newcommand{\dx}{\mathrm{d}x}
\newcommand{\ZZ}{\mathbb{Z}}
\newcommand{\cH}{\mathcal{H}}
\newcommand{\cL}{\mathcal{L}}

\newcommand{\cic}{\frac1{2\pi i}\int_C}
\renewcommand{\epsilon}{\varepsilon}
\renewcommand{\Re}{\operatorname{Re}}
\renewcommand{\Im}{\operatorname{Im}}
\DeclareMathOperator{\Log}{Log}
\DeclareMathOperator{\Arg}{Arg}
\DeclareMathOperator{\sign}{sign}
\DeclareMathOperator{\length}{length}
\DeclareMathOperator*{\Res}{Res}
\DeclareMathOperator{\ord}{ord}

\begin{document}

\section*{MATH 307 --- Worksheet \#9 }

\subsection*{Evaluating definite integrals using the residue theorem}
\bigskip
\noindent
\textbf{Theorem I.}
Let $R(x,y)$ be a rational function of $x$ and $y$ whose denominator doesn't vanish
on $|z|=1$. Then
\[
    \int_0^{2\pi} R(\cos\theta, \sin\theta)\,d\theta = 
    2\pi i \sum\{\text{residues of $f(z)$ inside $|z|=1$}\},
\]
where
\[
    f(z) = \frac1{iz}R\left(\frac12\left(z+\frac1z\right), \frac1{2i}\left(z - \frac1z\right)\right)
\]

\bigskip
\noindent
\textbf{Theorem II.}
\begin{enumerate}
\item Let $f(z)$ be analytic on the closed upper half-plane
\[\cH^*=\{z\in\CC : \Im z \geq 0\}\]
except for finitely many singularities in the open upper half-plane
\[\cH=\{z\in\CC : \Im z > 0\}.\]
Suppose there are positive constants $M$, $p$, and $R_0$ with $p>1$ such that
\[
|f(z)|\leq \frac{M}{z^p}\quad\text{for all $z\in \cH$ with $|z|\geq R_0$.}
\]
Then
\[
\int_{-\infty}^{\infty}f(x)\,dx = 2\pi i \sum\{\text{residues of $f(z)$ in $\cH$}\}.
\]
\item Let $f(z)$ be analytic on the closed lower half-plane
\[\cL^*=\{z\in\CC : \Im z \leq 0\}\]
except for finitely many singularities in the open lower half-plane
\[\cL=\{z\in\CC : \Im z < 0\}.\]
Suppose there are positive constants $M$, $p$, and $R_0$ with $p>1$ such that
\[
|f(z)|\leq \frac{M}{z^p}\quad\text{for all $z\in \cL$ with $|z|\geq R_0$.}
\]
Then
\[
    \int_{-\infty}^{\infty}f(x)\,dx = 2\pi i \sum\{\text{residues of $f(z)$ in $\cL$}\}.
\]
\item Both 1.{} and 2.{} hold when $f=P/Q$ is a rational function such that
\begin{enumerate}
    \item $\deg Q\geq \deg P + 2$, and
    \item $Q$ has no real roots.
\end{enumerate}

\end{enumerate}

\subsection*{Problems}

\begin{questions}
\setlength\itemsep{1em}
\setlength\parskip{1em}



\question
Evaluate the definite integral.

\begin{parts}
\part $\displaystyle I = \int_{-\infty}^\infty \frac{dx}{x^2-2x+4}$

\begin{solution}
    We have $$\deg (x^2-2x+4) = 2 \geq 2 + 0 = 2 + \deg 1.$$
    Moreover $x^2-2x+4$ has roots $\alpha$ and $\bar\alpha$,
    \[
        \alpha = \frac{2 + \sqrt{(-2)^2 - 4(1)(4)}}2 = 1+ \sqrt3 i,
    \]
    neither of which lie on the real line. Therefore, by Theorem II(3),
    \[
    I = 2\pi i\Res_{z=\alpha}\frac1{(z-\alpha)(z-\bar\alpha)}
    = \frac{2\pi i}{\alpha - \bar\alpha}
    = \frac{2\pi i}{2\sqrt 3 i}
    = \frac{\pi}{\sqrt3}.
    \]
\end{solution}

\part $\displaystyle I = \int_0^{2\pi} \frac{d\theta}{(5 - 3\sin\theta)^2}$

\begin{solution}
    Let $z=e^{i\theta}$, so that
    \[
        \sin\theta = \frac{z^2-1}{2iz}
        \quad\text{and}\quad d\theta = \frac{dz}{iz}.
    \]
    It follows that
    \begin{align*}
        I &= \int_{|z|=1}\frac{1}{\left(5 - 3\left(\frac{z^2-1}{2iz}\right)\right)^2}\frac{dz}{iz}\\
         &=  \int_{|z|=1}\frac{(2iz)^2}{\left(5(2iz) - 3\left(z^2-1\right)\right)^2}\frac{dz}{iz}\\
         &= \int_{|z|=1}\frac{4iz\,dz}{\left(3z^2 - 10iz - 3\right)^2}\\
         &= \frac{4i}9\int_{|z|=1}\frac{z\,dz}{(z-3i)^2(z-i/3)^2}
    \end{align*}
    The integrand has no singularities on $|z|=1$
    and its only singularity inside $|z|=1$ is a double pole at $z=i/3$.

    We have:
    \begin{align*}
        \Res_{z=i/3}\frac{z}{(z-3i)^2(z-i/3)^2} 
        &= \frac{d}{dz}\Big|_{z=i/3} (z-i/3)^2\frac{z}{(z-3i)^2(z-i/3)^2}\\
        &= \left.\frac{d}{dz}\right|_{z=i/3} \frac{z}{(z-3i)^2}\\
        &= \left.\frac{-z-3i}{(z-3i)^3}\right|_{z=i/3}\\
        &= -\frac{45}{256}
    \end{align*}
    Therefore,
    \[
        I = 2\pi i \cdot \frac{4i}9 \cdot \frac{-45}{256} = \frac{5\pi}{32}.
    \]
\end{solution}

\part $\displaystyle I = \int_0^{2\pi} \frac{d\theta}{a + b\sin\theta}\;$ where $a>|b|$
\begin{solution}
Let $z=e^{i\theta}$, so that
\[
    \sin\theta = \frac{z^2-1}{2iz}
    \quad\text{and}\quad d\theta = \frac{dz}{iz}.
\]
It follows that
\begin{align*}
    I &= \int_{|z|=1}\frac1{a + b\left(\frac{z^2-1}{2iz}\right)}\frac{dz}{iz}\\
    &= 2\int_{|z|=1}\frac{dz}{bz^2 + 2iaz - b}\\
    &= \frac2b\int_{|z|=1}\frac{dz}{(z-i\alpha)(z-i\beta)},
\end{align*}
\[\alpha = \frac{-a+\sqrt{a^2-b^2}}b,\quad \beta = \frac{-a-\sqrt{a^2-b^2}}b\]
Since $a>|b|$, $\beta < -1$ and $i\beta$ lies outside the closed unit disk.
Noting that $\alpha = 1/\beta$, it follows that $\alpha$ lies inside the open unit disk.
Therefore,
\begin{align*}
    I &= 2\pi i\cdot \frac2b\cdot \Res_{z=i\alpha}\frac1{(z-i\alpha)(z-i\beta)}\\
    &= 2\pi i\cdot \frac2b\cdot \frac1{i(\alpha - \beta)}\\
    &= 2\pi i\cdot \frac2b\cdot \frac{b}{2i\sqrt{a^2-b^2}}\\
    &= \frac{2\pi}{\sqrt{a^2-b^2}}.
\end{align*}
\end{solution}

\part $\displaystyle I = \int_{-\infty}^\infty \frac{1}{x^6+1}dx$

\begin{solution}
    Let $f(z)=1/(z^6+1)$. Then by Theorem II(3),
    \[I = 2\pi i \sum\{\text{residues of $f(z)$ in $\cH$}\}.\]
    The roots of the polynomial $z^6+1$ are
    \[z_k=e^{(\pi i + 2k\pi)/6},\quad k=0,1,2,3,4,5.\]
    The roots in the upper half-plane are
    \[
        z_0 = e^{\pi i/6},\quad z_1=e^{3\pi i/6},\quad\text{and}\quad
        z_2 = e^{5\pi i/6}.
    \]
    Being simple zeros of $z^6+1$, the points $z_k$ are
    simple poles of $f(z)$. Therefore,
    \begin{align*}
        \Res_{z=z_k}f(z) &= \lim_{z\to z_k}\frac{z-z_k}{z^6+1}\\
        &= \lim_{z\to z_k}\frac{1}{6z^5}\\
        &= \frac1{6z_k^5},
    \end{align*}
    by l'Hopital's rule.

    Thus,
    \begin{align*}
        \Res_{z=z_0}f(z) &= \frac16{e^{-5\pi i}}\\
        &= \frac16 \left(\frac{-\sqrt3 - i}2\right)\\
        &= -\frac{\sqrt3}{12} - \frac{i}{12}\\
        \Res_{z=z_1}f(z) &= \frac16e^{-15\pi i/6}\\
        &=\frac16e^{-\pi i/2}\\
        &= -\frac i6\\
        \Res_{z=z_0}f(z) &= \frac16e^{-25\pi i}\\
        &= \frac16e^{-\pi i/6}\\
        &= \frac{\sqrt3}{12} - \frac i{12}\\
    \end{align*}

    Therefore,
    \[
        I = 2\pi i\left(-\frac{\sqrt3}6 - \frac{i}{12} - \frac i6  + \frac{\sqrt3}{12} - \frac i{12}\right) = \frac{2\pi}3.
    \]
\end{solution}



\end{parts}
\end{questions}
\end{document}

\part $\displaystyle I = \int \,dx$

\begin{solution}
\end{solution}
